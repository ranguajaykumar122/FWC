\documentclass{article}
\usepackage{enumitem}
\usepackage{amssymb}
\usepackage{amsmath}
\usepackage{amsfonts}
\title{conic}
\begin{document}
\begin{enumerate}
\item The point at which the normal to the curve 
	\begin{align}
		y = x+ \frac{1}{x} \: ,x>0 
	\end{align}
	 is perpendicular to the line 
		\begin{align}
			3x-4y-7 = 0 
		\end{align}
	is:
\begin{enumerate}[label=(\alph*)]
\item $(2,\frac{5}{2})$
\item $(\pm{2},\frac{5}{2})$
\item $(\frac{-1}{2},\frac{5}{2})$
\item $(\frac{1}{2},\frac{5}{2}$
\end{enumerate}
\item The points on the curve 
	\begin{align}
		\frac{x^2}{9} + \frac{y^2}{16} = 1
	\end{align}
	at which the tangents are parallel to y-axis are:
	\begin{enumerate}[label=(\alph*)]
		\item $(0,\pm{4})$
		\item $(\pm{4},0)$
		\item $(\pm{3},0)$
		\item $(0,\pm{3})$
	\end{enumerate}
\item For which value of m is the line 
	\begin{align}
		y = mx +1
	\end{align}
	a tangent to the curve 
		\begin{align}
			y^2= 4x 
		\end{align}
	\begin{enumerate}[label=(\alph*)]
		\item $\frac{1}{2}$
		\item $1$
		\item $2$
		\item $3$
	\end{enumerate}
\end{enumerate}
\end{document}

